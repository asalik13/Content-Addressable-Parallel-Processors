\section{Introduction}
Content Addressable Parallel Processors (CAPPs) are an alternate architecture for memory. Unlike Random Access Memory (RAM) which works by 
performing operations for a word by using a memory address, a CAPP is able to select multiple memory addresses based on some information about the words. 
Furthermore, a CAPP can perform operations like multi-writing into and reading from multiple selected cells in constant time.
This is the reason CAPPs are so useful in routers and bridges as they are a much faster way to search for IP addresses to send packets to.
\\\\  
CAPPs are usually available in the form of Application Specific Integrated Chipsets (ASICs) and are used in routers, databases and more.
They can be divided into two broad categories, one being Binary CAPPs and the other Ternary. 
Ternary CAPPs come with the added flexibility of searching for words with masked bits, which is very similar to the regex for '.'. 
Of course, this comes with added circuity to each cell. 
\\\\ 
The goal of this project was to design a module for an expandable Ternary CAPP that is synthesizable on a FPGA using modern HDL tools. 
We also integrated a finite state machine that encapsulated the T-CAPP and a Serial UART module. 
This enabled the researchers to directly communicate with the T-CAPP from a CPU using serial communication.
The FSM enabled commands for reading, writing, selecting first as well as searching. 
\\\\
CAPPs have potential in graph theory, neural network caching, high-speed indexing, regex computations and a lot more. 
We hope that an open-source, expandable CAPP that can be synthesized on a FPGA can fill the gap for researchers to find new applications using an alternate memory architecture. 